\subsection*{運営方針}

\writtenBy{\president}{櫻井}{大輝}
%\writtenBy{\subPresident}{姓}{名}
%\writtenBy{\firstGrade}{姓}{名}
%\writtenBy{\secondGrade}{姓}{名}
%\writtenBy{\thirdGrade}{姓}{名}
%\writtenBy{\fourthGrade}{姓}{名}

\begin{itemize}
    \item 定例会議
    \item 局会議
    \item 局配属
    \item 企画
    
  \end{itemize}
  
\subsubsection*{定例会議}
会員全体で実施すべき議決は引き続き定例会議にて議決を行っていく.
週一回の開催を目処に行う.
定例会議の日程はアンケートで決める.
各局内での情報共有を増やす.

\subsubsection*{局会議}
局内の一部の人間のみが情報を把握している状況が起きないように局会議を活用していく.

\subsubsection*{局配属}
秋ごろに行う.
GW明け1か月程度で希望調査を行う.
局話し合いまでに配属を決定する.

\subsubsection*{企画}
MCPやMHPが伝統にすることもできるため新\secondGrade{}などで考えていく.


