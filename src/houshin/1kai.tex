\subsection*{\firstGrade{}方針}

\writtenBy{\firstGrade}{大村}{理駆}
%\writtenBy{\secondGrade}{姓}{名}
%\writtenBy{\thirdGrade}{姓}{名}
%\writtenBy{\fourthGrade}{姓}{名}

2024年度秋学期の\firstGrade{}方針について以下にまとめた.

\begin{itemize}
	\item 新規サービスの開発
	\item 勉強会の開催
	\item 入部勧誘
	\item 企業とのつながり
\end{itemize}

\subsubsection*{新規サービスの開発}

前学期では,\firstGrade{}は技術力向上を目指し,様々な分野の学習に励んできた.
次学期は,これらの学びを実践に繋げ,実用的なサービスの開発に注力する.
開発領域はWebサービスに留まらず,ゲーム,ライブラリ,ツール類など,部員の興味関心とスキルを最大限に活かせるよう,幅広い分野を視野に入れていく.
具体的なサービス内容は,部員間の議論を経て決定し,技術力向上のみならず,実践的な開発経験の蓄積,そしてポートフォリオの拡充に繋げる.
アウトプットを重視し,完成したサービスは積極的に公開・共有することで,サークル内外への貢献を目指していきたい.

\subsubsection*{勉強会の開催}

次学期は,新たにサークルへ参加する新入生を迎えるにあたり,スムーズなサークル活動への参加を支援するため,新入生向け勉強会を開催する.
勉強会では,プログラミング基礎や開発環境構築といった,サークル活動に必要な基礎知識の習得をサポートすると共に,既存部員との交流を深める機会を提供し,早期にサークルへ馴染めるよう工夫する.

\subsubsection*{入部勧誘}

次学期は,新入生歓迎会に加え,大学内での広報活動などを積極的に展開し,より多くの新入生に当サークルの魅力を伝え,入部を促する.
活発な活動内容や部員同士の友好性をアピールすることで,新入生に「このサークルで共に成長したい」と感じてもらえるよう,魅力的な勧誘活動を展開していく.

\subsubsection*{企業とのつながり}

前学期に引き続き,次学期も企業との連携を積極的に推進していく.
具体的には,勉強会教材の共同作成に加え,企業技術者による講演会や技術指導の実施,インターンシップ機会の獲得など,多岐にわたる連携を目指す.
将来的には,共同開発プロジェクトなども視野に入れ,より深い連携を目指していきたい.
