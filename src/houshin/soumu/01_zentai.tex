\subsection*{全体方針}

\writtenBy{\thirdGrade}{尾﨑}{真央}

\begin{itemize}
	\item 全体方針
	\item 風紀方針
	\item 掃除方針
	\item 書記方針
	\item 行事方針
	\item 備品方針
\end{itemize}

\subsubsection*{全体方針}
今年度もStudentActivityLoungeを使用できることが決まったため,共用スペースであることの自覚をもち,サークルルームの美化を徹底することが大事だと思われる.

\subsubsection*{風紀方針}
StudentActivityLoungeは食事禁止のため,それを重要視しながら「サークルームの美化」を徹底する.
BKCの部室では,しっかりゴミ箱に捨てるように徹底する.

\subsubsection*{掃除方針}
掃除業務に関しては,StudentActivityLoungeが共用スペースであるため,ゴミを見つけたら総務は捨てるように尽力する.BKC部室でも同様.

\subsubsection*{書記方針}
定例会議では総務部が議事録を記述する.
もし回せない場合は,discordでの必要な連絡を徹底する.

\subsubsection*{行事方針}
OIC移転により会内行事の見直しが必要になる.

\subsubsection*{備品業務方針}
サークルルームを使用できるため次のような方針で業務を行う.私物をサークルルームに置く人や本や機材を借りる人について総務局への連絡を必須とすることで総務局が状況を把握できるようにする. これにより延滞している場合,discordでの通知による返却催促を行う.
