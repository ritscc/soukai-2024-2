\subsection*{\firstGrade{}総括}

\writtenBy{\firstGrade}{大村}{理駆}
%\writtenBy{\secondGrade}{姓}{名}
%\writtenBy{\thirdGrade}{姓}{名}
%\writtenBy{\fourthGrade}{姓}{名}

2024年度秋学期の\firstGrade{}総括について以下にまとめた。

\begin{itemize}
	\item 新しい技術の習得
	\item 活発的な活動
	\item ハッカソンの参加
\end{itemize}

\subsubsection*{新しい技術の習得}

\firstGrade{}は、新しい技術を積極的に学ぶ機会を得た。
具体的には、プログラミングの基礎となる環境構築から始まり、Web開発(HTML, CSS, JavaScript, React, Next.js 等)、ゲーム開発(Unity, C# 等)、バックエンド開発(Rust, Axum, Go 等)といった幅広い分野の知識を習得した。

\subsubsection*{活発的な活動}

学期中は、主にDiscordを活用した活発なコミュニケーションを通じて、学習活動を推進した。
プログラミングに関する疑問点や課題に直面した際には、Discord上で積極的に質問や意見交換を行い、互いに協力しながら問題解決に取り組んだ。
また、各個人が興味を持つ技術や開発テーマについて自主的に学習を進め、成果を共有するなど、積極的な学習姿勢が全体に見られた。

\subsubsection*{ハッカソンの参加}
初めてハッカソンに参加し、共同開発のやり方や新しい技術についてより知識を深めることができた。
具体的にはプロジェクト管理のアプリケーションとGitHubの連携や、APIとAPI定義書、WebSocketの接続、などである。

\firstGrade{}として初めてハッカソンに参加し、チームでの共同開発を経験した。
ハッカソンでは、チーム内で協力し、アプリケーションの企画、開発、発表までの一連の流れを知ることができた。
具体的な技術としては、プロジェクト管理ツールとGitHubの連携による効率的なバージョン管理、APIとAPI定義書(OpenAPI)を用いたAPI連携、WebSocketによるリアルタイム通信、Next.js、SCSSによるデザイン、SSR周りの知識などを習得した。
