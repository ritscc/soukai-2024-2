\subsection*{2024年度秋学期総括}

\writtenBy{\president}{尾崎}{真央}
%\writtenBy{\subPresident}{姓}{名}
%\writtenBy{\firstGrade}{姓}{名}
%\writtenBy{\secondGrade}{姓}{名}
%\writtenBy{\thirdGrade}{姓}{名}
%\writtenBy{\fourthGrade}{姓}{名}

本会の目的である「情報科学の研究,及びその成果の発表を活動の基本に会員相互の親睦を図り,
学術文化の創造と発展に寄与する」ことを達成するため,方針として以下の五つを立てた.
これらについてそれぞれ評価を行うことで2024年度秋学期の総括とする.

\begin{itemize}
    \item 親睦を深める
    \item 規律ある行動
    \item 自己発信力の向上
    \item 会員間の技術向上
    \item 外部への情報発信
\end{itemize}

\subsubsection*{親睦を深める}
    2024年度秋学期活動では,主にプロジェクト活動を通して会員間の親睦を図った.どのプロジェクトでもオンラインやオフラインなどの活動形式を臨機応変に選択していた.

    また,3月にMonkeyConferencePartyを企画し,会員間だけでなく他大学との親睦を深めることを図った.

\subsubsection*{規律ある行動}
    2025年度秋学期の方針として,遅刻・欠席連絡と備品整理,
    サークルルームの使用方法の三つの項目からなる行動規範を定めた.

    遅刻・欠席連絡について,比較的開催時刻前の連絡を心がけることができていたと考える.

    サークルルームの使用方法については,2024年度から使用することになったStudentActivityLoungeにおいて,他団体との共有スペースとして利用している.

    食事禁止の規則に関しては徹底していた会員が多いように見受けられた.
    しかし,使用した備品を元に戻すという規則に関しては,モニターなどの位置を元に戻していない会員が見受けられた.

    学生部によるサークルルーム点検においても,特に戒告を受けることはなかった.


    備品整理については,基本的に備品を借りた会員がいなかったため,延滞注意などをすることがなかった,


\subsubsection*{自己発信力の向上}
    自己発信力を向上させるための機会として,2024年度秋学期活動では,
    LTを実施し,Advent CalendarはQiitaで実施した.

    定例会議におけるLTは,例年に比べてほとんどの学生がLTを行った.

\subsubsection*{会員間の技術向上}
    会全体の技術力向上について,\firstGrade{}が中心となってLTや作品開発を行っていたため,達成できたと考える.

    夏季休暇期間及び春季夏季休暇期間では,ハッカソンに参加する会員がいたため技術力向上につながった.

    また,3月にMonkeyConferencePartyを企画しているため,そこでの技術力向上も図れると考える.

\subsubsection*{外部への情報の発信}
    会外へ活動を発信する機会として,主に本会Webサイトと会公式Xが挙げられる.

    本会Webサイトはあまり更新しなかったが,会公式Xでは活動発信は頻繁に行えていたと考える.

    2024度は,Qiitaでorganizationを作成し,Advent CalendarもQiitaでの執筆を行った.

