\subsection*{運営総括}

%\writtenBy{\president}{}{}
\writtenBy{\subPresident}{長張}{快}
%\writtenBy{\firstGrade}{}{}
%\writtenBy{\secondGrade}{}{}
%\writtenBy{\thirdGrade}{}{}
%\writtenBy{\fourthGrade}{}{}

2024年度秋学期の運営を以下の4点から述べる.
\begin{itemize}
    \item 運営
    \item 定例会議
    \item 上回生会議
    \item 局
    \item 企画
\end{itemize}

\subsubsection*{運営}
2024年度秋学期の運営は,\thirdGrade{}が中心となって行った.
各局の仕事は\firstGrade{},\secondGrade{}が中心となって行い,\thirdGrade{}はサポートを行った.

\subsubsection*{定例会議}
毎週金曜日に定例会議を行った.
参加人数も10人以上は参加しており,特に\firstGrade{}の参加が目立った.
内容は例年通り執行部及び局からの連絡.会員によるLTであった.
2024年秋学期は\firstGrade{}がメインでLTを行った.
定例会議を開催する教室の予約は\kensuiChief{}が行った.

\subsubsection*{上回生会議}
随時議題が上がり次第.Discordを利用して開催した.
参加に関しては,参加できる上回生が参加するという形で行った.
内容としては,勉強会やLTを\firstGrade{}向けに調整,またイベントや産学連携についても議論を行った.
議事録は,Googleドライブに保存している.

\paragraph*{局会議}
局会議が開催された局は少なかったが.上回生会議で代わりの議論は行われていた.
局内での役割分担を明確にし,進捗を把握することが必要であると感じた.

\subsubsection*{企画}
企画の担当者は上回生で適宜担当者を変えつつ行った.
しかし,報連相の難易度が非常に高く,かなり進行難な場面が多かったため,対応が必要であると感じた.
対応策としては,役割分担を明確にし,進捗を把握することが必要であると感じた.
終了後にアンケートを実施し,その内容をもとにKPTを行った.
